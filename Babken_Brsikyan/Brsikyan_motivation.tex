\documentclass[11pt,a4paper,oneside]{report} % добавить leqno в [] для нумерации слева

%%% Работа с русским языком
\usepackage{cmap}					% поиск в PDF
\usepackage{mathtext} 				% русские буквы в формулах
\usepackage[T2A]{fontenc}			% кодировка
\usepackage[utf8]{inputenc}			% кодировка исходного текста
\usepackage[english,russian]{babel}	% локализация и переносы



%%% Дополнительная работа с математикой
\usepackage{amsmath,amsfonts,amssymb,amsthm,mathtools} % AMS
\usepackage{icomma} % "Умная" запятая: $0,2$ --- число, $0, 2$ --- перечисление

%% Номера формул
\mathtoolsset{showonlyrefs=true} % Показывать номера только у тех формул, на которые есть \eqref{} в тексте.

%% Шрифты
\usepackage{euscript}	 % Шрифт Евклид
\usepackage{mathrsfs} % Красивый матшрифт

%% Свои команды
\DeclareMathOperator{\sgn}{\mathop{sgn}}

%% Перенос знаков в формулах (по Львовскому)
\newcommand*{\hm}[1]{#1\nobreak\discretionary{}
{\hbox{$\mathsurround=0pt #1$}}{}}


\usepackage{setspace}
\onehalfspacing

\usepackage{pgf,tikz,pgfplots}
\pgfplotsset{compat=1.3}
\usepackage{mathrsfs}
\usetikzlibrary{arrows}

\begin{document}
	\begin{center}
		\textbf{\large Мотивационное письмо будущего (возможно) ассистента}
	\end{center} 
	\begin{flushleft}
		\begin{verse}
			"Деятельность человека бесплодна и ничтожна, когда не воодушевлена высокою идеею." 
		\end{verse}
	\end{flushleft}
    \begin{flushright}
		$  \textcopyright$ Чернышевский Н.Г.
    \end{flushright}

Свое мотивационное письмо хотелось бы начать с того, что Теория вероятностей для меня является безумно увлекательным предметом, которым постоянно хочется делиться с окружающими людьми. Этот предмет является интересным и применимым на практике во всех сферах деятельности. Как говорил мне отец: "Сынок, учи тервер, он пригодится".

Будучи опытным человеком в провождениях консультаций (проводил полгода консультации по линейной алгебре), могу уверенно сказать, что данной деятельностью мне очень нравится заниматься и я готов тратить много времени на нее. 

Теория вероятностей - необходимый предмет для моей деятельности и ассистенство не даст мне забыть о нем. Таким образом, я помогу и себе, и людям, которым буду проводить консультации. 

Раз уж начал свое мотивационное письмо с цитаты, предпочитаю и закончить цитатой моего Благородного наставника:
	\begin{flushleft}
	\begin{verse}
		"Бабкен, ты похож чем-то на меня. У тебя все получится!" 
	\end{verse}
	\end{flushleft}
	\begin{flushright}
	$  \textcopyright$ Зехов Матвей
	\end{flushright}

	\begin{center}
		\textbf{\large Мои Данные}
	\end{center} 
	\begin{enumerate}
		\item Брсикян Бабкен Араевич
		\item Группа Бэк 171
		\item Оценка: 8
		\item Владею языками программирования R, Python на хорошем уровне. Умею работать на tex.
		\item Телефон: 8(926)725-01-11\\
		Почта: brsikyanbabken99@gmail.com\\
		Телеграм канал: @brsikyanbabken\\
		Номер галактики в Тентуре не имею, потому что живу в Антитентуре
		
	\end{enumerate}

\end{document}