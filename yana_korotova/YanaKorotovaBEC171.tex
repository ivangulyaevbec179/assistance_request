\documentclass[a4paper, 12pt]{scrreprt} 

    \usepackage{cmap}
    \usepackage{graphicx} 
    \usepackage{wrapfig}
    \usepackage{lastpage}
    \usepackage{tikz}
    \usepackage{pgfplots}
    \usepackage{braket}
    \usepackage{verbatim}
    \usetikzlibrary{arrows}
    \usetikzlibrary{calc,positioning,fit,backgrounds}
    \usetikzlibrary{decorations.pathreplacing,calc}
    \usepackage{amsmath} 
    \usepackage{color}
    \usepackage[T2A]{fontenc}
    \usepackage[utf8]{inputenc}	
    \usepackage[english,russian]{babel}
    \usepackage{geometry} 
    \geometry{top=25mm}
    \geometry{bottom=25mm}
    \geometry{left=16mm}
    \geometry{right=16mm}	
    \usepackage{amssymb}
    \usepackage{icomma} 
    \usepackage{mathtext} 
    \usepackage{mathrsfs}
    \usepackage{mathtools}
    \usepackage{fancyhdr}
    \usepackage{mdframed}
    \newmdenv[
  topline=false,
  bottomline=false,
  skipabove=\topsep,
  skipbelow=\topsep,
  leftmargin=20pt,
  rightmargin=20pt,
  innertopmargin=0pt,
  innerbottommargin=0pt
]{siderules}
    %\pagestyle{fancy}
    %\fancyhf{}
    %\renewcommand{\headrulewidth}{0,04mm}
    \usepackage{mathtext} 
    \usepackage{multicol}
    %\rhead{\scshape{BEC 171}}
    %\lhead{\scshape{Research seminar for research stream}}
    %\chead{} 
    \cfoot{\thepage} 
    
    \makeatletter 
    \renewcommand{\headrulewidth}{0,4mm} 
    %ЦВЕТА
    \newcommand{\blue}[1]{\textcolor[rgb]{.3,.3,.5}{#1}}
    \newcommand{\ot}[1]{\textcolor[rgb]{.55,.55,.65}{#1}}
    \newcommand{\oq}[1]{\textcolor[rgb]{.6,.6,.6}{#1}}

    \renewcommand{\maketitle}{\noindent{\bfseries\scshape\LARGE\blue\@title}\par
        \noindent {\large\scshape\bfseries\@subtitle}\par
        \noindent {\slshape\mdseries\oq\@author}
        \vskip 2ex}

    \renewcommand\section{\@startsection{section}{1}{\z@}%
        {-3.5ex \@plus -1ex \@minus -.2ex}%
        {-1em}%
        {\normalfont\large\slshape\bfseries\ot}}
    \makeatother

    \usepackage{color,hyperref}
\definecolor{darkblue}{rgb}{0.0,0.0,0.3}
\hypersetup{colorlinks,breaklinks,
            linkcolor=darkblue,urlcolor=darkblue,
            anchorcolor=darkblue,citecolor=darkblue}

    \newcommand\doilink[1]{\href{http://dx.doi.org/#1}{#1}}
    \newcommand\doi[1]{doi:\doilink{#1}}
    \providecommand*\url[1]{\href{#1}{#1}}
    \renewcommand*\url[1]{\href{#1}{\texttt{#1}}}
    \providecommand*\email[1]{\href{mailto:#1}{#1}}
        
    \renewcommand{\labelitemi}{$\diamond$}

    \author{Ассистенство на самом крутом курсе :)}
    \title{Мотивационное письмо}

\begin{document}
\maketitle
\pagestyle{empty} 

\section*{Почему я хочу быть ассистентом на курсе по теории вероятностей и математической статистике?}~\

Несколько доводов:
\begin{itemize}
    \item Во-первых, как мне кажется, это лучший математический курс на нашей программе. Нас учили не заучивать алгоритмы, чтобы решать простейшие задачи, а понимать, что происходит в этом мире. 
    \item Во-вторых, мне понравилось быть учебным ассистентом на <<Микроэкономике -- 1>> для исследовательского потока. Я проверяла домашние работы, составляла задачи, которые в них входили, помогала наблюдать за первокурсниками во время написания промежуточной контрольной.
    \item В-третьих, команда преподавателей по теории вероятностей кажется наиболее привлекательной с точки зрения работы с ними. 
\end{itemize}

\section*{А почему именно тервер, а не микра?}~\

Мне понравилось взаимодействие между студентами, ассистентами и преподавателями. До безумия я полюбила теорию вероятностей! :) В первую очередь хочется помочь замечательным преподавателям в организации, проведении контрольных, в каком-то смысле отблагодарить их за то, что дисциплина была такой замечательной!

Помимо этого, мне нравится объяснять людям разный материал, преподавать, а это входит в обязанности учебных ассистентов по терверу. Особенно сильно мне нравится объяснять математику!

\section*{Про меня :)}~\

Вообще меня многие называют лягушкой (один мой одногруппник), но у меня есть имя! Настоящее, человеческое.

\begin{center}
    \begin{tabular}{ll}
        \hline
        \hline
        Имя, отчество, фамилия: & \href{https://vk.com/mmmlsl}{Яна Николаевна Коротова} \\
        Группа: & БЭК 171 \\
        Оценка по теории вероятностей: & 10 \\
        \hline
        R: & не знаю, но регрессии, например, умею строить \\
        Python: & прослушала несколько курсов на DataCamp \\
        Latex: & вроде нормасик \\
        \hline
        Номер телефона: & +7 (977) 346 42 86 \\
        E-mail: & \email{yanankorotova@gmail.com} \\
        Telegram: & \href{https://t.me/mmmlsl}{mmmlsl} \\
        \hline
        \hline
    \end{tabular}
\end{center}

В заключение хочу поделиться с прекрасным благороднейшим Доном Пуассоном ссылкой на одно \href{https://rutube.ru/video/12d4c39005a0f29901c25aa28122db3c/}{видео}, смотреть которое я никак не могу перестать :)

\end{document}
